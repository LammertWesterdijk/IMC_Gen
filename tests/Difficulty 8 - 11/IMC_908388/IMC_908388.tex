\documentclass[12pt]{article}
\usepackage{amssymb, amsmath}
\usepackage{enumerate, enumitem}
\usepackage{pgf,tikz}
\usepackage{subcaption}
\usetikzlibrary{arrows}

\usepackage[top=1.5cm,bottom=2cm,left=2cm,right=2cm]{geometry}

\setlength{\parindent}{0pt}
\setlength{\parskip}{\baselineskip}

\newtheorem{opg}{Problem}

\renewcommand{\labelenumi}{\alph{enumi})}

\newcommand{\opp}[1]{\text{\normalfont{op}}( \triangle #1)}


\begin{document}

\parbox{0.6\textwidth}{\center {\Huge Practice IMC} \\[2ex] {\Large Seed: 908388}}
\parbox{0.4\textwidth}{\includegraphics[width=6cm]{logo.png}}

\hrule

\begin{opg}
Let  $n$  be a positive integer. Starting with the sequence
$1, \frac{1}{2}, \frac{1}{3}, \dots, \frac{1}{n}$,
form a new sequence of  $n-1$  entries
$\frac{3}{4}, \frac{5}{12}, \dots, \frac{2n-1}{2n(n-1)}$
by taking the averages of
two consecutive entries in the first sequence. Repeat the
averaging of neighbors on the second sequence to obtain a third
sequence of  $n-2$  entries, and continue until the final sequence produced
consists of a single number  $x_n$.  Show that  $x_n < 2/n$.

\end{opg}
\begin{opg}
A blackboard contains 68 pairs of nonzero integers.  Suppose that for each positive integer $k$ at most one of the pairs $(k, k)$ and $(-k, -k)$ is written on the blackboard.  A student erases some of the 136 integers, subject to the condition that no two erased integers may add to 0.  The student then scores one point for each of the 68 pairs in which at least one integer is erased.  Determine, with proof, the largest number $N$ of points that the student can guarantee to score regardless of which 68 pairs have been written on the board.


\end{opg}
\begin{opg}
Let $n \ge 3$ be an integer, and consider a circle with $n + 1$ equally spaced points marked on it. Consider all labellings of these points with the numbers $0, 1, ... , n$ such that each label is used exactly once; two such labellings are considered to be the same if one can be obtained from the other by a rotation of the circle. A labelling is called beautiful if, for any four labels $a < b < c < d$ with $a + d = b + c$, the chord joining the points labelled $a$ and $d$ does not intersect the chord joining the points labelled $b$ and $c$.

Let $M$ be the number of beautiful labelings, and let N be the number of ordered pairs $(x, y)$ of positive integers such that $x + y \le n$ and $\gcd(x, y) = 1$. Prove that \[M = N + 1.\]


\end{opg}
\begin{opg}
A set of lines in the plane is in $\textit{general position}$ if no two are parallel and no three pass through the same point. A set of lines in general position cuts the plane into regions, some of which have finite area; we call these its $\textit{finite regions}$. Prove that for all sufficiently large $n$, in any set of $n$ lines in general position it is possible to colour at least $\sqrt{n}$ of the lines blue in such a way that none of its finite regions has a completely blue boundary.


\end{opg}
\begin{opg}
Let $n$ be a positive integer. What is the largest $k$ for which there exist $n \times n$ matrices $M_1, \dots, M_k$ and $N_1, \dots, N_k$ with real entries such that for all $i$ and $j$, the matrix product $M_i N_j$ has a zero entry somewhere on its diagonal if and only if $i \neq j$?

\end{opg}


\end{document}