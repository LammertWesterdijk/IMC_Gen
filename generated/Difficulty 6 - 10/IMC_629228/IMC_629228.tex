\documentclass[12pt]{article}
\usepackage{amssymb, amsmath}
\usepackage{enumerate, enumitem}
\usepackage{pgf,tikz}
\usepackage{subcaption}
\usetikzlibrary{arrows}

\usepackage[top=1.5cm,bottom=2cm,left=2cm,right=2cm]{geometry}

\setlength{\parindent}{0pt}
\setlength{\parskip}{\baselineskip}

\newtheorem{opg}{Problem}

\renewcommand{\labelenumi}{\alph{enumi})}

\newcommand{\opp}[1]{\text{\normalfont{op}}( \triangle #1)}


\begin{document}

\parbox{0.6\textwidth}{\center {\Huge Practice IMC} \\[2ex] {\Large Seed: 629228}}
\parbox{0.4\textwidth}{\includegraphics[width=6cm]{logo.png}}

\hrule

\begin{opg}
A triangle is called a parabolic triangle if its vertices lie on a
parabola $y = x^2$. Prove that for every nonnegative integer $n$, there
is an odd number $m$ and a parabolic triangle with vertices at three
distinct points with integer coordinates with area $(2^nm)^2$.


\end{opg}
\begin{opg}
A computer screen shows a $98 \times 98$ chessboard, colored in the usual way. One can select with a mouse any rectangle with sides on the lines of the chessboard and click the mouse button: as a result,  the colors in the selected rectangle switch (black becomes white, white becomes black). Find, with proof, the minimum number of mouse clicks needed to make the chessboard all one color.


\end{opg}
\begin{opg}
(a) Prove that if the six dihedral (i.e. angles between pairs of faces) of a given tetrahedron are congruent, then the tetrahedron is regular.

(b) Is a tetrahedron necessarily regular if five dihedral angles are congruent?


\end{opg}
\begin{opg}
Determine all functions $f$ from the set of positive integers to the set of positive integers such that, for all positive integers $a$ and $b$, there exists a non-degenerate triangle with sides of lengths 


 $a,f(b)$ and $f(b+f(a)-1)$. 

(A triangle is non-degenerate if its vertices are not collinear.) 

Author: Bruno Le Floch, France


\end{opg}
\begin{opg}
For each positive integer $k$, let $A(k)$ be the number of odd divisors of $k$ in the interval $[1, \sqrt{2k})$. Evaluate
\[
\sum_{k=1}^\infty (-1)^{k-1} \frac{A(k)}{k}.
\]

\end{opg}


\end{document}